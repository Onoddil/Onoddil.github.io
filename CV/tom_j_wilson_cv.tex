%%%%%%%%%%%%%%%%%%%%%%%%%%%%%%%%%%%%%%%%%
% Medium Length Graduate Curriculum Vitae
% LaTeX Template
% Version 1.1 (9/12/12)
%
% This template has been downloaded from:
% http://www.LaTeXTemplates.com
%
% Original author:
% Rensselaer Polytechnic Institute (http://www.rpi.edu/dept/arc/training/latex/resumes/)
%
% Important note:
% This template requires the res.cls file to be in the same directory as the
% .tex file. The res.cls file provides the resume style used for structuring the
% document.
%
%%%%%%%%%%%%%%%%%%%%%%%%%%%%%%%%%%%%%%%%%

%----------------------------------------------------------------------------------------
%	PACKAGES AND OTHER DOCUMENT CONFIGURATIONS
%----------------------------------------------------------------------------------------

\documentclass[letter, margin, 10pt]{res} % Use the res.cls style, the font size can be changed to 11pt or 12pt here
\usepackage[linkcolor=black,colorlinks=true,urlcolor=cyan]{hyperref}
\usepackage{helvet} % Default font is the helvetica postscript font
%\usepackage{newcent} % To change the default font to the new century schoolbook postscript font uncomment this line and comment the one above

\topmargin=-0.3in
\oddsidemargin -.4in
\resumewidth = 6.9in
\addtolength{\textheight}{1.2in}

\setlength{\textwidth}{5.5in} % Text width of the document

\begin{document}

%----------------------------------------------------------------------------------------
%	NAME AND ADDRESS SECTION
%----------------------------------------------------------------------------------------

\moveleft.5\hoffset\centerline{\large\bf Tom J. Wilson} % Your name at the top
 
\moveleft\hoffset\vbox{\hrule width\resumewidth height 1pt}\smallskip % Horizontal line after name; adjust line thickness by changing the '1pt'
 
\moveleft.5\hoffset\centerline{Space Telescope Science Institute} % Your address
\moveleft.5\hoffset\centerline{3700 San Martin Drive, Baltimore, MD 21218}
\moveleft.5\hoffset\centerline{towilson@stsci.edu, \href{https://orcid.org/0000-0001-6352-9735}{ORCID: 0000-0001-6352-9735}}
\moveleft.5\hoffset\centerline{\href{https://onoddil.github.io}{onoddil.github.io}, \href{https://www.twitter.com/onoddil}{@Onoddil}}

%----------------------------------------------------------------------------------------
\begin{resume}

%----------------------------------------------------------------------------------------
%	OBJECTIVE SECTION
%----------------------------------------------------------------------------------------
 
% \section{RESEARCH INTERESTS}  

% My main interest in astrophysics involves the understanding of early star formation, from cloud core through to cluster evolution along the pre-main-sequence. Specifically, the application of robust statistical methodology to investigations to explore and explain these complex astrophysical processes. Understanding the consequences of the way we observe the universe and characterising how our assumptions influence our conclusions also drive my interests in the field.

%----------------------------------------------------------------------------------------
%	EDUCATION SECTION
%----------------------------------------------------------------------------------------

\section{PROFESSIONAL WORK}

{\sl Postdoctoral Researcher}, Space Telescope Science Institute, 2018-\\
Analysis and optimisation of supernovae survey for future WFIRST observing strategies. Involves solving maximisation problem of information content analysis, and robust statistical analysis of the fitting and modelling of supernovae lightcurves. Software development for \texttt{photutils}, an \texttt{astropy}-affiliated python package, focussing on analysis tools for PSF photometry. Requires code testing and improvements, package maintenance and distribution, and documentation development.

\section{EDUCATION}

{\sl PhD in Physics}, Exeter, 2013-2018\\
Main focus of thesis surrounds the effects of blended stars on the astrometric and photometric properties of stars in crowded Galactic fields. Characterisation of photometric catalogues by angular resolution and survey completeness explored, allowing for quantitative comparisons of the crowding suffered. A probability-based Bayesian cross-matching process was constructed, allowing for robust determination of detections across catalogues. Analysed the effect crowding has on the \textit{WISE} catalogue. \\Awarded STFC PhD Studentship. \\Supervisor: Prof. Tim Naylor

{\sl MPhys in Physics with Astrophysics,} Master's Dissertation, Exeter, 2012-2013 \\
Analysis of young stellar cluster ages using the radiative-convective gap as a distance and extinction independent variable. Subsequent analysis of the initial mass function to explore effects of age and distance on masses in stellar clusters. Characterised the effect the choice of magnitudes used in a colour-colour diagram has on the interstellar extinction-age dependency, determining the optimal CCD for age independent extinction derivations for low mass young pre-main-sequence stars. \\Supervisor: Prof. Tim Naylor

{\sl College Summer Internship}, Exeter, 2012\\
Six week undergraduate work experience during undergraduate degree. Initially involved improvements to data reduction for JCMT 450$\mu$m and 850$\mu$m observations. Subsequent work on the construction of temperature and dust $\beta$ maps from these observations, supplemented with \textit{Herschel} data in several cases, concluding with analysis of NGC1333.\\Supervisor: Dr Jennifer Hatchell

{\sl MPhys in Physics with Astrophysics,} First Class Honours, Exeter, 2009-2013 \\
University of Exeter, Undergraduate degree

%----------------------------------------------------------------------------------------
%	PUBLICATION SECTION
%----------------------------------------------------------------------------------------
\parskip 5pt
\section{FIRST AUTHOR PUBLICATIONS}

Wilson Tom J., Naylor T., 2018, MNRAS, 481, 2148; ``A Contaminant-Free Catalogue of \textit{Gaia} DR2-\textit{WISE} Galactic Plane Matches: Including the Effects of Crowding in the Cross-Matching of Photometric Catalogues''

Wilson Tom J., Naylor T., 2018, MNRAS, 473, 5570; ``Improving Catalogue Matching By Supplementing Astrometry with Additional Photometric Information''

Wilson Tom J., Naylor T., 2017, MNRAS, 469, 2517; ``The Effect of Unresolved Contaminant Stars on the Cross-matching of Photometric Catalogues''

\section{CO-AUTHOR PUBLICATIONS}
  
Wakeford H. R., Wilson T. J., et al., 2019, RNAAS, 3, 7; ``Exoplanet Atmosphere Forecast: Observers Should Expect Spectroscopic Transmission Features to be Muted to 33\%''

Wakeford H. R., Lewis N. K., Fowler J., Bruno G., Wilson T. J., et al., 2019, AJ, 157, 11; ``Disentangling the Planet from the Star in Late-Type M Dwarfs: A Case Study of TRAPPIST-1g''

Wakeford H. R., Sing D. K., Deming D., Lewis N. K., Goyal J., Wilson T. J., et al., 2018, AJ, 155, 29; ``The Complete Transmission Spectrum of WASP-39b with a Precise Water Constraint''

Rumble D., Hatchell J., Pattle K., Kirk H., Wilson T., et al., 2016, MNRAS, 460, 4150; ``The JCMT Gould Belt Survey: Evidence for Radiative Heating and Contamination in the W40 Complex''

Rees J., Wilson T., et al., 2016, IAUS, 314, 205; ``The Age of Taurus: Environmental Effects on Disc Lifetimes''

Hatchell J., Wilson T., et al., 2013, MNRAS, 429, 10; ``The JCMT Gould Belt Survey: SCUBA-2 Observations of Radiative Feedback in NGC 1333''

% Rees J., Wilson Tom J., et al., in prep.; ``Pre-main-sequence Isochrones -- IV. the Age of Taurus and Increased Disc Lifetimes in Low-Density Environments''

%----------------------------------------------------------------------------------------
%	TALKS SECTION
%----------------------------------------------------------------------------------------
\parskip \baselineskip
\section{SCIENTIFIC TALKS \& CONFERENCES}

July 2019, Python in Astronomy 19, Contributed Talk\\
June 2019, STScI, HotSci@STScI Colloquia, Contributed Seminar Talk\\
March 2019, STScI, Friday Science Coffee, Contributed Seminar Talk\\
February 2019, STScI, TESS Data Workshop, Contributed Talk\\
February 2019, UNLV BUFFALO 2019 Meeting, Contributed Talk\\
May 2018, Exeter, First Year PhD Development Day, Invited Talk\\
March 2018, Science with Precision Astrometry, Contributed Poster\\
September 2017, Cardiff Star Formation Workshop, Contributed Talk\\
July 2016, NASA Goddard Space Flight Center, Invited Seminar Talk\\
June 2016, Cool Stars 19, Contributed Poster\\
April 2015, BECSS Bristol, Contributed Talk\\
March 2015, Milky Way Astrophysics from Wide-Field Surveys, Contributed Talk\\

%----------------------------------------------------------------------------------------
%	PROFESSIONAL EXPERIENCE SECTION
%----------------------------------------------------------------------------------------
 
\section{TEACHING, OUTREACH \& EXPERIENCE}

{\sl Undergraduate Astrophysics Lab Demonstrator}, 2013-2017\\
Assist in teaching of 20-30 second year undergraduates in the astrophysics portion of the teaching lab. Includes demonstrating use of UNIX commands, IRAF, data reduction and analysis, as well as report writing. Additional duties included report assessment and feedback. Experiments included analysis of Cepheid variability periods, colour-magnitude diagrams, and optical spectroscopy.

{\sl Undergraduate Astrophysics Teaching Telescope Operator}, 2013-2017\\
Lead undergraduate students in usage of the university's teaching telescope. Teaching of telescope operation, as well as remote/automated capacities provided by ACP Scheduler and understanding of observation planning.

{\sl Undergraduate Physics Problem Tutor}, 2013-2017\\
Assist in running of first year undergraduate physics problems classes. Involved running homework sessions, managing approximately 140 students. Provided guidance on a wide variety of problems, ranging from solid state physics to electromagnetism to astrophysics. 

{\sl ``Pint of Science'' Event Organiser}, 2013-2018\\
Co-organised the ``Pint of Science'' outreach events held in Exeter each year. Events involve three nights of outreach talks from a selection of researchers, aimed at public engagement. Role included a host of responsibilities including deciding the themes for each nights' talks, inviting speakers, and organising event location. Role also included engaging with the community and answering questions from members of the public.

{\sl Observing At the James Clerk Maxwell Telescope, Hawaii}, 2014\\
Experience at the JCMT as part of the JPS survey. Involved telescope operation, data processing, and observation scheduling.

 
%----------------------------------------------------------------------------------------
%	COMPUTER SKILLS SECTION
%----------------------------------------------------------------------------------------

\section{COMPUTER \\ SKILLS}

Developer for AstroPy Photutils Package

Python (NumPy, SciPy, AstroPy, MatPlotLib), Fortran (OPENMP), StarLink, Topcat, IRAF, ACP Scheduler, Maxim, DS9, \LaTeX, OSX, Linux, Windows

%----------------------------------------------------------------------------------------
%	REFERENCES SECTION
%----------------------------------------------------------------------------------------

% \section{REFERENCES} 

% Dr Louis-Gregory Strolger, Space Telescope Science Institute, strolger@stsci.edu

% Prof. Tim Naylor, University of Exeter, timn@astro.ex.ac.uk

% Dr Jennifer Hatchell, University of Exeter, hatchell@astro.ex.ac.uk

 
----------------------------------------------------------------------------------------

\end{resume}
\end{document}




