%%%%%%%%%%%%%%%%%%%%%%%%%%%%%%%%%%%%%%%%%
% Medium Length Graduate Curriculum Vitae
% LaTeX Template
% Version 1.1 (9/12/12)
%
% This template has been downloaded from:
% http://www.LaTeXTemplates.com
%
% Original author:
% Rensselaer Polytechnic Institute (http://www.rpi.edu/dept/arc/training/latex/resumes/)
%
% Important note:
% This template requires the res.cls file to be in the same directory as the
% .tex file. The res.cls file provides the resume style used for structuring the
% document.
%
%%%%%%%%%%%%%%%%%%%%%%%%%%%%%%%%%%%%%%%%%

%----------------------------------------------------------------------------------------
%	PACKAGES AND OTHER DOCUMENT CONFIGURATIONS
%----------------------------------------------------------------------------------------

\documentclass[letter, margin, 10pt]{res} % Use the res.cls style, the font size can be changed to 11pt or 12pt here
\usepackage[linkcolor=black,colorlinks=true,urlcolor=cyan]{hyperref}
\usepackage{helvet} % Default font is the helvetica postscript font
%\usepackage{newcent} % To change the default font to the new century schoolbook postscript font uncomment this line and comment the one above
\usepackage{enumitem}

\topmargin=-0.7in
\oddsidemargin -.6in
\resumewidth = 6.9in
\addtolength{\textheight}{1.2in}

\setlength{\textwidth}{6.1in} % Text width of the document

\begin{document}

%----------------------------------------------------------------------------------------
%	NAME AND ADDRESS SECTION
%----------------------------------------------------------------------------------------

\moveleft.5\hoffset\centerline{\large\bf Tom J. Wilson} % Your name at the top
 
\moveleft\hoffset\vbox{\hrule width\resumewidth height 1pt}\smallskip % Horizontal line after name; adjust line thickness by changing the '1pt'
 
\moveleft.5\hoffset\centerline{University of Exeter} % Your address
\moveleft.5\hoffset\centerline{Physics Building, Stocker Road, Exeter, EX4 4QL}
\moveleft.5\hoffset\centerline{t.j.wilson@exeter.ac.uk / onoddil@pm.me, \href{https://orcid.org/0000-0001-6352-9735}{ORCID: 0000-0001-6352-9735}}
\moveleft.5\hoffset\centerline{\href{https://onoddil.github.io}{onoddil.github.io}, \href{https://www.twitter.com/onoddil}{@Onoddil} --- Updated August 2023}

%----------------------------------------------------------------------------------------
\begin{resume}

%----------------------------------------------------------------------------------------
%	OBJECTIVE SECTION
%----------------------------------------------------------------------------------------
 
% \section{RESEARCH INTERESTS}  

% My main interest in astrophysics involves the understanding of early star formation, from cloud core through to cluster evolution along the pre-main-sequence. Specifically, the application of robust statistical methodology to investigations to explore and explain these complex astrophysical processes. Understanding the consequences of the way we observe the universe and characterising how our assumptions influence our conclusions also drive my interests in the field.

%----------------------------------------------------------------------------------------
%	EDUCATION SECTION
%----------------------------------------------------------------------------------------
\vspace{-9pt}
\section{PROFESSIONAL WORK}

{\sl Postdoctoral Research Fellow}, University of Exeter, 2019-\\
\null\quad\quad Awarded two STFC grants
\begin{itemize}[noitemsep,topsep=0pt,parsep=0pt,partopsep=0pt]
\item Creation of software pipeline for cross-matching external photometric catalogues to LSST datasets as part of LSST:UK consortium, writing tens of thousands of lines of code across multiple programming languages
\item Efficient analysis of datasets of 10s of billions of objects, ensuring statistical robustness, correcting for systematic effects and correlations
\item Working within the Vera C. Rubin Observatory commissioning team on verifying and validating telescope astrometry, and ensuring that astrometric uncertainties produced by the LSST pipeline match observations
\end{itemize}
\vspace{-10pt}
{\sl Postdoctoral Researcher}, Space Telescope Science Institute, 2018-2019\\
\null\quad\quad STScI Director's Discretionary Research Fund
\begin{itemize}[noitemsep,topsep=0pt,parsep=0pt,partopsep=0pt]
\item Software development for \texttt{photutils}, an \texttt{astropy}-affiliated python package, focussing on analysis tools for PSF photometry
\item Developed optimised observing strategy for \textit{NASA}'s \$3 billion flagship spacecraft, the \textit{Roman} Space Telescope
\item Involved solving maximisation problem of information content analysis, and robust statistical analysis of the fitting and modelling of supernovae lightcurves
\item Differential analysis of wide-field \textit{Hubble} datasets, improving search efficiency of transient events
\end{itemize}
\vspace{-6pt}
\section{EDUCATION}

{\sl PhD in Physics}, University of Exeter, 2013-2018 --- Supervisor: Prof. Tim Naylor\\
\null\quad\quad Awarded STFC PhD Studentship
\begin{itemize}[noitemsep,topsep=0pt,parsep=0pt,partopsep=0pt]
\item Developed modern statistical methods for solving catalogue cross-matching problem in astronomy, optimised for datasets of billions of objects
\item Complex characterisation of datasets in various correlated parameters, allowing for quantitative comparisons across independent datasets
\end{itemize}
\vspace{-10pt}
{\sl Master's Dissertation}, University of Exeter, 2012-2013 --- Supervisor: Prof. Tim Naylor
\begin{itemize}[noitemsep,topsep=0pt,parsep=0pt,partopsep=0pt]
\item Analysis of datasets for previously unexplored parameterisations of data in colour-magnitude diagrams to enable new solutions for outstanding problems
\item Exploration of parameter phase space and determination of key impacting factors in the initial mass function of stellar clusters
\item Developed novel analysis tools to disentangle correlated parameters, resolving degeneracies within multi-dimensional datasets, breaking degeneracies with age in colour-colour diagrams
\end{itemize}
\vspace{-10pt}
{\sl College Summer Internship}, University of Exeter, 2012 --- Supervisor: Dr Jennifer Hatchell\\
\null\quad\quad Awarded University College Studentship
\begin{itemize}[noitemsep,topsep=0pt,parsep=0pt,partopsep=0pt]
\item Work on the construction of temperature and dust $\beta$ maps from JCMT 450$\mu$m and 850$\mu$m observations, supplementing existing datasets with new data and updated analysis methodology
\end{itemize}
\vspace{-10pt}
{\sl MPhys in Physics with Astrophysics,} First Class Honours, University of Exeter, 2009-2013

%----------------------------------------------------------------------------------------
%	COMPUTER SKILLS SECTION
%----------------------------------------------------------------------------------------
\vspace{-6pt}
\section{SOFTWARE ENGINEERING}

Developer for the \texttt{astropy} \texttt{photutils} package
\begin{itemize}[noitemsep,topsep=0pt,parsep=0pt,partopsep=0pt]
\item 8th highest contribution (2nd highest 2018-2020) to the community-driven, open source software
\item Code tests, feature improvements, package maintenance, and documentation development
\end{itemize}
\vspace{-10pt}
Lead developer of the \texttt{macauff} python package
\begin{itemize}[noitemsep,topsep=0pt,parsep=0pt,partopsep=0pt]
\item Development of computationally efficient and precise codebase to ensure accurate photometry and astrometry can be obtained for a set of cross-matched photometric catalogues
\item Writing entire project end-to-end, including creation of continuous integration suite, documentation, and class-based python and fortran code
\end{itemize}
\vspace{-10pt}
Other project highlights:
\begin{itemize}[noitemsep,topsep=0pt,parsep=0pt,partopsep=0pt]
\item Optimisation and efficiency strategy for \textit{Roman} Space Telescope's supernovae survey, impacting two years of flagship space telescope observations
\item Improvements to data reduction for JCMT observations, and writing of key data analysis pipeline
\end{itemize}
\vspace{-10pt}
Main programming strengths: python (\texttt{astropy}), fortran (\texttt{openmp}), statistical analysis and application

%----------------------------------------------------------------------------------------
%	PUBLICATION SECTION
%----------------------------------------------------------------------------------------
\parskip 5pt
\vspace{-6pt}
\section{GRANTS AND AWARDS}
2023-2025, Co-PI STFC Grant, "UK Involvement in LSST: Phase C (Exeter Component)", £310,000\\
2022, ``Above and Beyond'' Award, University of Exeter -- extending second-year astrophysics lab experiments and converting from IRAF to Python\\
2019-2023, Co-writer and lead researcher STFC Grant, ``UK Involvement in LSST: Phase B (Exeter Component)'', £438,000

\parskip 5pt
\vspace{-2pt}
\section{FIRST AUTHOR PUBLICATIONS}
Wilson Tom J., Naylor T., in prep.; ``Improvements to the Astrometric Uncertainty Function: Updated Algorithms for Astrometric Perturbations due to Unresolved Contaminants in Background-Dominated Sources and More Accurate Flux Contaminations''\\
Wilson Tom J., 2023, RASTI, 2, 1; ``Overcoming Separation Between Counterparts Due to Unknown Proper Motions in Catalogue Cross-Matching''\\
Wilson Tom J., 2022, RNAAS, 6, 60; ``A Parameterized Model for Differential Galaxy Counts at Any Wavelength''\\
Wilson Tom J., 2021, RNAAS, 5, 265; ``On the Use of Evidence and Goodness-of-Fit Metrics in Exoplanet Atmosphere Interpretation''\\
Wilson Tom J., Naylor T., 2018, MNRAS, 481, 2148; ``A Contaminant-Free Catalogue of \textit{Gaia} DR2-\textit{WISE} Galactic Plane Matches: Including the Effects of Crowding in the Cross-Matching of Photometric Catalogues''\vspace{2pt}\\
Wilson Tom J., Naylor T., 2018, MNRAS, 473, 5570; ``Improving Catalogue Matching By Supplementing Astrometry with Additional Photometric Information''\vspace{2pt}\\
Wilson Tom J., Naylor T., 2017, MNRAS, 468, 2517; ``The Effect of Unresolved Contaminant Stars on the Cross-matching of Photometric Catalogues''
\vspace{-5pt}
\section{CO-AUTHOR PUBLICATIONS}
Wilson A. J., Lakeland B. S., Wilson T. J., Naylor T., 2023, MNRAS, 521, 354; ``A Naive Bayes Classifier for identifying Class II YSOs''\\
Astropy Collaboration, ..., Wilson T. J., et al., 2022, ApJ, 935, 167; ``The Astropy Project: Sustaining and Growing a Community-oriented Open-source Project and the Latest Major Release (v5.0) of the Core Package''\\
Bruno G., Lewis N. K., Valenti J. A., Pagano I., Wilson T. J., et al., 2022, MNRAS, 509, 5030; ``Hiding in Plain Sight: Observing Planet-Starspot Crossings with the \textit{James Webb Space Telescope}''\\
Bradley L., Sip\H{o}cz B., Robitaille T., Tollerud E., Vinícius Z., Deil C., Barbary K., Wilson T. J., et al., 2019-2022, 10.5281/zenodo.7419741; ``astropy/photutils: v0.6 through v1.6.0''\vspace{2pt}\\
Lewis N. K., ..., Wilson T. J., et al., 2020, ApJL, 902, 19; ``Into the UV: The Atmosphere of the Hot Jupiter HAT-P-41b Revealed''\vspace{2pt}\\
Wakeford H. R., Sing D. K., Stevenson K. B., Lewis N. K., Pirzkal N., Wilson T. J., et al., 2020, AJ, 159, 204; ``Into the UV: A Precise Transmission Spectrum of HAT-P-41b Using Hubble's WFC3/UVIS G280 Grism''\vspace{2pt}\\
Steinhardt C., ..., Wilson T. J., et al., 2020, ApJS, 247, 64; ``The Buffalo HST Survey''\vspace{2pt}\\
Wakeford H. R., Wilson T. J., et al., 2019, RNAAS, 3, 7; ``Exoplanet Atmosphere Forecast: Observers Should Expect Spectroscopic Transmission Features to be Muted to 33\%''\vspace{2pt}\\
Wakeford H. R., Lewis N. K., Fowler J., Bruno G., Wilson T. J., et al., 2019, AJ, 157, 11; ``Disentangling the Planet from the Star in Late-Type M Dwarfs: A Case Study of TRAPPIST-1g''\vspace{2pt}\\
Wakeford H. R., Sing D. K., Deming D., Lewis N. K., Goyal J., Wilson T. J., et al., 2018, AJ, 155, 29; ``The Complete Transmission Spectrum of WASP-39b with a Precise Water Constraint''\vspace{2pt}\\
Rumble D., Hatchell J., Pattle K., Kirk H., Wilson T., et al., 2016, MNRAS, 460, 4150; ``The JCMT Gould Belt Survey: Evidence for Radiative Heating and Contamination in the W40 Complex''\vspace{2pt}\\
Rees J., Wilson T., et al., 2016, IAUS, 314, 205; ``The Age of Taurus: Environmental Effects on Disc Lifetimes''\vspace{2pt}\\
Hatchell J., Wilson T., et al., 2013, MNRAS, 429, 10; ``The JCMT Gould Belt Survey: SCUBA-2 Observations of Radiative Feedback in NGC 1333''

% Rees J., Wilson Tom J., et al., in prep.; ``Pre-main-sequence Isochrones -- IV. the Age of Taurus and Increased Disc Lifetimes in Low-Density Environments''

%----------------------------------------------------------------------------------------
%	TALKS SECTION
%----------------------------------------------------------------------------------------
\parskip \baselineskip
\vspace{-6pt}
\section{SCIENTIFIC TALKS \& CONFERENCES}
4 invited seminars, 4 invited conference talks,\\
4 contributed seminars, 19 contributed conference talks, 5 contributed conference posters.

\vspace{-4pt}
August 2023, Rubin Project and Community Workshop 2023, Contributed Talk\\
July 2023, National Astronomy Meeting 2023, UK Involvement in LSST, Contributed Talk\\
July 2023, National Astronomy Meeting 2023, 10 years after Herschel, Contributed Talk\\
July 2023, National Astronomy Meeting 2023, Euclid exploitation in the UK, Contributed Talk\\
March 2023, 3rd TVS Software Workshop, Invited Talk\\
October 2022, LSST@Europe4, Contributed Talk\\
July 2022, National Astronomy Meeting 2022, Contributed Talk\\
July 2022, Cool Stars 21, Contributed Poster\\
June 2022, Asteroseismology MW-Gaia Workshop, Contributed Talk\\
June 2022, Bristol-Cardiff Joint Seminar Series, Invited Seminar\\
May 2022, Carnegie EPL, Invited Seminar\\
February 2022, University of Delaware, Invited Seminar\\
August 2021, Rubin Observatory Project and Community Workshop, Contributed Talk\\
July 2021, National Astronomy Meeting 2021, Contributed Talk\\
July 2021, National Astronomy Meeting 2021, Invited Talk\\
June 2021, Statistical Challenges in Modern Astronomy VII, Contributed Poster\\
May 2021, LSST:UK All-Hands Meeting, Invited Talk\\
May 2021, LSST:UK All-Hands Meeting, Contributed Talk\\
April 2021, UKEXOM 2021, Contributed Talk\\
April 2021, University of Exeter, Contributed Seminar\\
March 2021, Cool Stars 20.5, Contributed Poster\\
October 2020, RAS Specialist Meeting: TVS with Rubin Observatory, Contributed Talk\\
August 2020, Rubin Observatory Project and Community Workshop, Contributed Talk\\
July 2020, University of Exeter, Contributed Seminar\\
July 2019, Python in Astronomy 19, Contributed Talk\\
June 2019, STScI, HotSci@STScI Colloquia, Contributed Seminar\\
March 2019, STScI, Friday Science Coffee, Contributed Seminar\\
February 2019, STScI, TESS Data Workshop, Contributed Talk\\
February 2019, UNLV BUFFALO 2019 Meeting, Contributed Talk\\
May 2018, Exeter, First Year PhD Development Day, Invited Talk\\
March 2018, Science with Precision Astrometry, Contributed Poster\\
September 2017, Cardiff Star Formation Workshop, Contributed Talk\\
July 2016, NASA Goddard Space Flight Center, Invited Seminar\\
June 2016, Cool Stars 19, Contributed Poster\\
April 2015, BECSS Bristol, Contributed Talk\\
March 2015, Milky Way Astrophysics from Wide-Field Surveys, Contributed Talk

%----------------------------------------------------------------------------------------
%	PROFESSIONAL EXPERIENCE SECTION
%----------------------------------------------------------------------------------------
 \vspace{-6pt}
\section{COMMITTEE \& SERVICE}

August 2023, Rubin Project and Community Workshop 2023 Crowded Fields Session, Science Organising Committee\\
May 2021, LSST:UK All-Hands Meeting, Local Organising Committee\\
June 2019, Hubble Support Scientist for Cycle 29 Time Allocation Committee\\
Peer review referee for AAS Journals and JOSS\\
Member of LSST Stars, Milky Way and Local Volume Science Collaboration (SMWLVSC)\\
Member of LSST SMWLVSC Crowded Stellar Field Task Force\\
Member of Vera C. Rubin Observatory SIT-Com Commissioning team (Astrometric Science Unit)\\

 \vspace{-6pt}
\section{TEACHING \& OUTREACH EXPERIENCE}

{\sl Astrophysics Lab Re-Write Lead}, 2021-
\begin{itemize}[noitemsep,topsep=0pt,parsep=0pt,partopsep=0pt]
\item Spearhead updating of second year astrophysics lab course content
\item Involves translating materials from IRAF to Python
\item Update pedagogy to ensure that students are engaging with the material in the most effective way
\item Review course aims and opportunities for students to practice experimental analysis and programming skills developed in complementary modules
\end{itemize}

\vspace{-10pt}

{\sl Undergraduate Astrophysics Lab Demonstrator}, 2013-2022
\begin{itemize}[noitemsep,topsep=0pt,parsep=0pt,partopsep=0pt]
\item Assist in teaching of 20-30 second year undergraduates in the astrophysics portion of the lab
\item Demonstrated use of UNIX commands, IRAF and Python, data reduction and analysis, \& report writing
\item Duties included report assessment and feedback
\item Experiments included analysis of Cepheid variability periods, colour-magnitude diagrams, and optical spectroscopy
\end{itemize}

\vspace{-10pt}

{\sl Undergraduate Python Demonstrator}, 2021
\begin{itemize}[noitemsep,topsep=0pt,parsep=0pt,partopsep=0pt]
\item Teaching Python to approximately 30 second year undergraduates
\item Demonstrating key Python and general programming skills, logical thinking, and problem solving
\item Includes live assistance, technical support, and formal work marking and feedback
\end{itemize}

\vspace{-10pt}

{\sl Undergraduate Astrophysics Teaching Telescope Operator}, 2013-2017
\begin{itemize}[noitemsep,topsep=0pt,parsep=0pt,partopsep=0pt]
\item Lead undergraduate students in usage of the university's teaching telescope
\item Teaching of telescope operation, as well as remote/automated capacities of ACP Scheduler
\item Involved teaching an understanding of observation planning
\end{itemize}

\vspace{-10pt}

{\sl Undergraduate Physics Problem Tutor}, 2013-2017
\begin{itemize}[noitemsep,topsep=0pt,parsep=0pt,partopsep=0pt]
\item Assist in running first year undergraduate physics problems classes
\item Involved organising homework sessions, managing approximately 140 students
\item Provided guidance on a wide variety of problems, ranging from solid state physics to electromagnetism to astrophysics
\end{itemize}

\vspace{-10pt}

{\sl ``Pint of Science'' Event Organiser}, 2013-2018
\begin{itemize}[noitemsep,topsep=0pt,parsep=0pt,partopsep=0pt]
\item Co-organised the ``Pint of Science'' outreach events held in Exeter each year
\item Events involve three nights of outreach talks from a selection of researchers, aimed at public engagement
\item Included a host of responsibilities including deciding the themes for each nights' talks, inviting speakers, and organising event location
\item Involved engaging with the community and answering questions from members of the public
\end{itemize}

\vspace{-10pt}

{\sl Observing At the James Clerk Maxwell Telescope, Hawaii}, 2014
\begin{itemize}[noitemsep,topsep=0pt,parsep=0pt,partopsep=0pt]
\item Experience at the JCMT as part of the JPS survey
\item Involved telescope operation, data processing, and observation scheduling
\end{itemize}

% Awards - summer placement, 2012 ~£1500
 
%----------------------------------------------------------------------------------------
%	REFERENCES SECTION
%----------------------------------------------------------------------------------------

% \section{REFERENCES} 

% Dr Louis-Gregory Strolger, Space Telescope Science Institute, strolger@stsci.edu

% Prof. Tim Naylor, University of Exeter, timn@astro.ex.ac.uk

% Dr Jennifer Hatchell, University of Exeter, hatchell@astro.ex.ac.uk

 

\end{resume}
\end{document}




